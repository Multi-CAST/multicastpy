%%%%%%%%%%%%%%%%%%%%%%%%%%%%%%%%%%%%%%%%%%%%%%%%%%%%%%%%%%%%%%%%%%
%%%%%%%%%%%%%%%%%%%%%%%%%%%%%%%%%%%%%%%%%%%%%%%%%%%%%%%%%%%%%%%%%%

\section{Corpus languages}
\label{sec:corpora}

% UPDATE %
\sloppypar As of \getlist[2]{vkey}, the Multi-CAST collection comprises data from \theCmetacorpora\ languages: Arta, Bora, \mbox{Cypriot} Greek, English, Jinghpaw, Kalamang, Mandarin, Matukar Panau, Nafsan, Northern \mbox{Kurdish}, Persian, \mbox{Sanzhi} Dargwa, Sumbawa, \mbox{Tabasaran}, Teop, Tondano, Tulil, and Vera'a. It encompasses \num{\theCmetatexts} individual texts and roughly \ovhours\ hours of recordings, \ovclauses\ clause units, and \ovwords\ words. Each corpus in the collection is treated as its own contribution, and is hence an individually citable resource with the annotators as authors. %  Chirag

This section provides a brief outline of the various Multi-CAST corpora. \refx{tab:statistics} summarizes selected corpus statistics, and the map in \refx{fig:locatormap} offers a geographical overview of the included languages. Comprehensive metadata on the texts and speakers can be found in Appendices~\ref{apx:list-of-texts} and \ref{apx:list-of-speakers}.
%
The \sqt{glottocodes} listed below reference entries in the \tit{Glottolog} \pcite{Hammarstrom.etal2021}.\footnote{\turl{https://glottolog.org/}{glottolog.org/}} \sqt{Identifiers} are the corpus labels used internally in Multi-CAST and \tit{multicastR}. For an explanation of the versioning system used by Multi-CAST, see \refx{ssec:versioning}.

% UPDATE %
%%%%%%%%%%
\input{graphics/mc_locatormap.tex}
%%%%%%%%%%

% UPDATE %
%%%%%%%%%%
\input{tables/mc_overview-table.tex}
%%%%%%%%%%


% ------------------------------------------------------------------------------------------------------------------------------------------------------------------------------ %
% ------------------------------------------------------------------------------------------------------------------------------------------------------------------------------ %

\subsection{Arta}
\label{ssec:corpus-arta}

\noindent\tit{Yukinori Kimoto}

\begin{description}[labelwidth=6.5em,itemindent=0em,itemsep=0.25mm]
	\TabPositions{2em}
	\raggedright\small
	\item[glottocode]		\texttt{arta1239}
	\item[affiliation]		Austronesian, Malayo-Polynesian, Northern Luzon
	\item[area spoken]		the Philippines, Luzon, Quirino Province
	\item[varieties rec'd]	Arta
	\item[text types]		traditional narratives,\\autobiographical narratives
	\item[sources]		\bcite{Kimoto2017}, \ycite{Kimoto2018}
	\medskip
	\item[identifier]		\texttt{arta}
	\item[availability]		since August 2019, version \texttt{1908}
	\item[GRAID]		7.0		\tab	{\small(\texttt{\tgeq\ths1908})}
	\item[RefIND]		\checkno{}	%\tab	{\small(\texttt{\tgeq\ths1908})}
	\item[ISNRef]		\checkno{}	%\tab	{\small(\texttt{\tgeq\ths1908})}
	\item[citation]		\hyperref[ssec:references-mc]{Kimoto, Yukinori. 2019. Multi-CAST Arta. In Haig, Geoffrey \& Schnell, Stefan (eds.), \tit{Multi-CAST: Multilingual corpus of annotated spoken texts}. (\turl{}{multicast.aspra.uni-bamberg.de/\#arta})} \nocite{Kimoto2019}
\end{description}

\noindent Arta is a critically endangered Austronesian language spoken by a group of hunter-gatherers living in Luzon, the Philippines. The number of fluent speakers is between nine and eleven, most of which are over the age of forty. Since all speakers have settled down in the communities of neighbouring Negrito groups (Casiguran/Nagitupunan Agta people), the language is not in active use and no longer taught to children. All of the speakers are multilingual with Casiguran/Nagtipunan Agta and Ilokano.

The texts in this corpus were collected by Yukinori Kimoto during fieldwork in the Quirino and Aurora provinces in Luzon between 2012 and 2018. See \tcite{Kimoto2017} for a description of the language.


% - - - - - - - - - - - - - - - - - - - - - - - - - - - - - - - - - - - - - - - - - - - - - - - - - - - - - - - - - - - - - - - - - - - - - - - - - - - - - - - - - - - - - - - - - - - - - - - - - - - - %
% - - - - - - - - - - - - - - - - - - - - - - - - - - - - - - - - - - - - - - - - - - - - - - - - - - - - - - - - - - - - - - - - - - - - - - - - - - - - - - - - - - - - - - - - - - - - - - - - - - - - %

\subsubsection*{Background to the recordings}

\paragraph{alisiya}		% 0001
Speaker AR01. The speaker talks about how she fell ill when she was young, and how her illness and the lifelong paralysis that resulted from have affected her life.

\paragraph{arsenyo}	% 0614
Speaker AR02. The speaker talks about his best friend Arsenyo (AR03), telling an impressive story about him, who, among others, took care of the speaker during their hunting trips together.

\paragraph{child}		% 0100
Speaker AR03. The speaker talks about the difficulties he and his wife faced raising their children as a result of poverty, lack of schooling, and insufficient medical care. 

\paragraph{delia}		% 0601
Speaker AR01. An autobiography. The speaker tells stories about how badly he behaved when he was young, about how he married the present wife, and about the influence of religious missionaries.

\paragraph{disubu}		% 0002
Speaker AR03. A description of the food the speaker and his contemporaries used to eat in their childhood. It also includes a description about the different activities conducted by men and women in their own hunting and gathering societies.

\paragraph{hapon}		% 0117
Speaker AR02. The speaker shares his father's stories of the hardships endured during Japanese occupation of the Philippines and the Pacific War, when the Arta people were forced to hide in the forests near their villages for fear of their lives.

\paragraph{husband}	% 0502
Speaker AR01. The speaker talks about her late husband, telling several stories about him, including one involving the New People's Army.

\paragraph{marry}		% 0564
Speaker AR02. A message to a newly married couple. The speaker speaks about the social norms they should follow, and advises them to always be considerate of each other.

\paragraph{swateng}	% 0632
Speaker AR03. A folk story about a man called Sanuwateng, who came to the lowlands to marry an Arta girl. Because of his prolonged absence following his courtship, she decides to marry another man, which leads to a tragic and bloody ending.

\paragraph{typhoon}	% 0007
Speaker AR01. A narrative about the typhoon that hit the Arta community in August 2013. The speaker is telling how the whole community dealt with the natural disaster during and after the typhoon.

\paragraph{udulan}	% 0106
Speaker AR03. Two short folk stories about two men: Udulan is the main character of the story of a marriage between two different Negrito groups from the eastern and western sides of Sierra Madre, and Sanuwateng is the villain of a tragic story of intertribal marriage, the longer version of which is found in the text \tit{swateng}.


% ------------------------------------------------------------------------------------------------------------------------------------------------------------------------------ %
% ------------------------------------------------------------------------------------------------------------------------------------------------------------------------------ %

\subsection{Bora}
\label{ssec:corpus-bora}

\noindent\tit{Frank Seifart, Tai Hong}

\begin{description}[labelwidth=6.5em,itemindent=0em,itemsep=0.25mm]
	\TabPositions{2em}
	\raggedright\small
	\item[glottocode]		\texttt{bora1263}
	\item[affiliation]		Boran
	\item[area spoken]		southern Colombia, northern Peru
	\item[varieties rec'd]	Bora
	\item[text types]		traditional narratives
%	\item[sources]		\bcite{((???))}
	\medskip
	\item[identifier]		\texttt{bora}
	\item[availability]		since July 2022, version \texttt{2207}
	\item[GRAID]		8.0		\tab	{\small(\texttt{\tgeq\ths2022})}
	\item[RefIND]		\checkyes{}	\tab	{\small(\texttt{\tgeq\ths2022})}
	\item[ISNRef]		\checkyes{}	\tab	{\small(\texttt{\tgeq\ths2022})}
	\item[citation]		\hyperref[ssec:references-mc]{Seifart, Frank \& Hong, Tai. 2022. Multi-CAST Bora. In Haig, Geoffrey \& Schnell, Stefan (eds.), \tit{Multi-CAST: Multilingual corpus of annotated spoken texts}. (\turl{}{multicast.aspra.uni-bamberg.de/\#tabasaran})} \nocite{Seifart.Hong2022}
\end{description}

\noindent Bora is a Boran language spoken in various small communities in the Colombian and Peruvian Amazon region (e.g.\@ 3.23°S 71.99°W, 1.75°S 72.50°W). The language has approximately \num{1000} speakers, almost all of whom are bilingual in local Spanish. The number of children acquiring Bora is currently decreasing.

Bora has been extensively documented within a VolkswagenStiftung-funded DOBES documentation project (2005\tnd2009).\footnote{\url{https://hdl.handle.net/1839/42550453-b3db-4d83-b30f-3bce5304588e}} The Multi-CAST Bora corpus consists of two folkloristic narrative texts taken from the larger DOBES collection. They were recorded and annotated by Frank Seifart in collaboration with, especially, Clever Panduro (original transcription and translation) and Lena Sell (original morphological glossing). Annotations with GRAID and RefIND were added to the corpus in 2021\tnd2022 by Tai Hong in collaboration with Frank Seifart.
%
Session name correspondences between Multi-CAST and the DOBES archive are as follows:
%
\begin{itemize}
	\TabPositions{7em}
	\item	\tit{ajyuwa}		\tab	\tit{piivyeebe\_ajyu}
	\item	\tit{meenujkatsi}	\tab	\tit{meenujkatsi}
\end{itemize}


% ------------------------------------------------------------------------------------------------------------------------------------------------------------------------------ %
% ------------------------------------------------------------------------------------------------------------------------------------------------------------------------------ %

\subsection{Cypriot Greek}
\label{ssec:corpus-cypgreek}

\noindent\tit{Harris Hadjidas, Maria Vollmer}

\begin{description}[labelwidth=6.5em,itemindent=0em,itemsep=0.25mm]
	\TabPositions{2em}
	\raggedright\small
	\item[glottocode]		\texttt{cypr1249}
	\item[affiliation]		Indo-European, Greek, Attic
	\item[area spoken]		Cyprus
	\item[varieties rec'd]	Yeri-Pyroi
	\item[text types]		traditional narratives
	\medskip
	\item[identifier]		\texttt{cypgreek}
	\item[availability]		since May 2015, version \texttt{1505}
	\item[GRAID]		7.0		\tab	{\small(\texttt{\tgeq\ths1505})}
	\item[RefIND]		\checkyes{}	\tab	{\small(\texttt{\tgeq\ths1905})}
	\item[ISNRef]		\checkyes{}	\tab	{\small(\texttt{\tgeq\ths1905})}
	\item[citation]		\hyperref[ssec:references-mc]{Hadjidas, Harris \& Vollmer, Maria C. 2015. Multi-CAST Cypriot Greek. In Haig, Geoffrey \& Schnell, Stefan (eds.), \tit{Multi-CAST: Multilingual corpus of annotated spoken texts}. (\tcode{multicast.aspra.uni-bamberg.de/\#cypgreek})} \nocite{Hadjidas.Vollmer2015}
\end{description}

\noindent Cypriot Greek is the variety of Greek spoken in Cyprus. The three texts in this subcorpus, all of which are traditional narratives, were originally recorded in the 1960s, and later compiled and published by Konstantinos Giangoullis as part of a book of traditional Cypriot tales \pcite{Giangoullis2009}:
%
\begin{itemize}
	\TabPositions{5em}
	\item	\tit{jitros}	\tab	from pages 51\tnd53,
	\item	\tit{minaes}	\tab	from pages 47\tnd51, and
	\item	\tit{psarin}	\tab	from pages 84\tnd88.
\end{itemize}
%
The speaker in these texts, Elenis Mich (CG01), grew up and spent her life in the village of Yeri-Pyroi, near Nicosia.
%
Unfortunately, no recordings are available for the texts. They appear to have been only minimally edited, and reflect reasonably faithfully the spoken language used in traditional narratives. The author of the text collection, Konstantinos Giangoullis, has kindly given his permission for the three texts to be made freely available in Multi-CAST. % We refer to the author's website for further information on his work and related publications.\footnote{Previously online at~~\url{www.cypriot-folk-poets.com/}.}

The texts were originally transliterated into the roman alphabet and translated into English by a native speaker, Harris Hadjidas, who also conducted an initial round of syntactic annotation with an earlier version of GRAID. A second round of annotation, adhering to the guidelines of the GRAID 7.0, was completed by Maria Vollmer under supervision of Geoffrey Haig.


% ------------------------------------------------------------------------------------------------------------------------------------------------------------------------------ %
% ------------------------------------------------------------------------------------------------------------------------------------------------------------------------------ %

\subsection{English}
\label{ssec:corpus-english}

\noindent\tit{Nils Norman Schiborr}

\begin{description}[labelwidth=6.5em,itemindent=0em,itemsep=0.25mm]
	\TabPositions{2em}
	\raggedright\small
	\item[glottocode]		\texttt{sout3282}
	\item[affiliation]		Indo-European, Germanic, West
	\item[area spoken]		United Kingdom
	\item[varieties rec'd]	Southeast and South England
	\item[text types]		autobiographical narratives
	\item[sources]		\bcite{Huddleston.Pullum2002}
	\medskip
	\item[identifier]		\texttt{english}
	\item[availability]		since May 2015, version \texttt{1505}
	\item[GRAID]		7.0		\tab	{\small(\texttt{\tgeq\ths1505})}
	\item[RefIND]		\checksome{}	\tab	{\small(\texttt{\tgeq\ths1908})}
	\item[ISNRef]		\checksome{}	\tab	{\small(\texttt{\tgeq\ths1908})}
	\item[citation]		\hyperref[ssec:references-mc]{Schiborr, Nils N. 2015. Multi-CAST English. In Haig, Geoffrey \& Schnell, Stefan (eds.), \tit{Multi-CAST: Multilingual corpus of annotated spoken texts}. (\turl{}{multicast.aspra.uni-bamberg.de/\#english})} \nocite{Schiborr2015}
\end{description}

\noindent The Multi-CAST English corpus contains autobiographical narratives taken from the Freiburg English Dialect Corpus (FRED, English Dialects Research Group 2005),\footnote{Note that the audio recordings in this corpus are in the public domain, and thus do not fall under the \tit{Creative Commons} licence applied to the annotations and the rest of Multi-CAST.} which has been compiled under the supervision of Bernd Kortmann and Lieselotte Anderwald at the University of Freiburg from texts recorded during the 1970s and 80s as part of various oral history projects. Session name correspondences between Multi-CAST and FRED are as follows:
%
\begin{itemize}
	\TabPositions{5em}
	\item	\tit{devon01}	\tab	\tit{DEV\_002}
	\item	\tit{kent01}		\tab	\tit{KEN\_002}
	\item	\tit{kent02}		\tab	\tit{KEN\_002}
	\item	\tit{kent03}		\tab	\tit{KEN\_004}
	\item	\tit{london01}	\tab	\tit{LND\_006}, \tit{LND\_007}
\end{itemize}
%
The texts annotated for Multi-CAST were recorded with older working-class speakers from southern and southeastern England. They depict everyday scenes and personal experiences from the speakers' lives: recurring topics include agriculture, animal husbandry, shipwrighting, work in the London docks, and the two World Wars.


% - - - - - - - - - - - - - - - - - - - - - - - - - - - - - - - - - - - - - - - - - - - - - - - - - - - - - - - - - - - - - - - - - - - - - - - - - - - - - - - - - - - - - - - - - - - - - - - - - - - - %
% - - - - - - - - - - - - - - - - - - - - - - - - - - - - - - - - - - - - - - - - - - - - - - - - - - - - - - - - - - - - - - - - - - - - - - - - - - - - - - - - - - - - - - - - - - - - - - - - - - - - %

%\subsubsection*{Background to the recordings}
%
%\paragraph{kent01}
%
%
%
%two recording sessions with the same speaker (EN01), two weeks apart.
%
%\tedit{corresponds to file xyz in FRED}.
%
%
%\paragraph{kent02}


% ------------------------------------------------------------------------------------------------------------------------------------------------------------------------------ %
% ------------------------------------------------------------------------------------------------------------------------------------------------------------------------------ %

\subsection{Jinghpaw}
\label{ssec:corpus-jinghpaw}

\noindent\tit{Keita Kurabe}

\begin{description}[labelwidth=6.5em,itemindent=0em,itemsep=0.25mm]
	\TabPositions{2em}
	\raggedright\small
	\item[glottocode]		\texttt{kach1280}
	\item[affiliation]		Tibeto-Burman, Sal
	\item[area spoken]		Kachin State, Myanmar; India; People's Republic of China
	\item[varieties rec'd]	Myitkyina
	\item[text types]		traditional narratives
	\item[sources]		\bcite{Kurabe2016}, \ycite{Kurabe2012}, \ycite{Kurabe2018}
	\medskip
	\item[identifier]		\texttt{jinghpaw}
	\item[availability]		since June 2021, version \texttt{2106}
	\item[GRAID]		7.0		\tab	{\small(\texttt{\tgeq\ths2106})}
	\item[RefIND]		\checkyes{}	\tab	{\small(\texttt{\tgeq\ths2106})}
	\item[ISNRef]		\checkyes{}	\tab	{\small(\texttt{\tgeq\ths2106})}
	\item[citation]		\hyperref[ssec:references-mc]{Kurabe, Keita. 2021. Multi-CAST Jinghpaw. In Haig, Geoffrey \& Schnell, Stefan (eds.), \tit{Multi-CAST: Multilingual corpus of annotated spoken texts}. (\turl{}{multicast.aspra.uni-bamberg.de/\#jinghpaw})} \nocite{Kurabe2021}
\end{description}

\noindent Jinghpaw, also known as Kachin, is a Tibeto-Burman language spoken in Myanmar and adjacent areas of China and India. The variety represented in the corpus is spoken in and around Myitkyina, Kachin State, Myanmar. The Jinghpaw speakers, as is typical for highlanders in mainland Southeast Asia, live in a socioculturally dynamic and multilingual environment. Of particular importance is the fact that Jinghpaw serves as a lingua franca among the Kachin people, who are a linguistically diverse people speaking many mutually unintelligible Tibeto-Burman languages, but who have a number of shared cultural traits.

The Multi-CAST Jinghpaw corpus consists of traditional narratives glossed and annotated by Keita Kurabe with the help of Stefan Schnell. They constitute a subset of more than \num{2700} traditional Kachin narratives and related stories told in Jinghpaw, which were collected by Keita Kurabe and members from the Kachin community through a community-based documentation project undertaken in northern Myanmar between 2009 and 2020. Audio recordings for \num{2754} stories with \num{1751} transcriptions are currently archived in PARADISEC (\bcite{Kurabe2013}, \ycite{Kurabe2017}).\footnote{\turl{http://catalog.paradisec.org.au/collections/KK1}{catalog.paradisec.org.au/collections/KK1}}\footnote{\turl{http://catalog.paradisec.org.au/collections/KK2}{catalog.paradisec.org.au/collections/KK2}} Session name correspondences with Multi-CAST are as follows:
%
\begin{itemize}
	\TabPositions{5em}
	\item	\tit{chyeju}		\tab	\tit{0276}	\quad	\sqt{The wolf and the water bird}		% Htu Bu	2017	JG01
	\item	\tit{dwi}		\tab	\tit{0269}	\quad	\sqt{The orphan and his grandmother}		% Htu Bu	2017
	\item	\tit{galang}		\tab	\tit{0274}	\quad	\sqt{The man who became a mad vulture}	% Htu Bu	2017
	\item	\tit{ganu}		\tab	\tit{0187}	\quad	\sqt{The widow's son}				% M. Ji Nan	2017	JG02
	\item	\tit{hkaili}		\tab	\tit{0262}	\quad	\sqt{The man who married a bad wife}		% Htu Bu	2017
	\item	\tit{hpaji}		\tab	\tit{0275}	\quad	\sqt{The wolf and the crow}			% Htu Bu	2017
	\item	\tit{manau}		\tab	\tit{1861}	\quad	\sqt{The haughty Indian night jar}		% M. Tu Ja	2015	JG03
%	\item	\tit{myit}		\tab	\tit{0265}	\quad	\sqt{The honest man}				% Htu Bu	2017
	\item	\tit{natga}		\tab	\tit{0319}	\quad	\sqt{The woman who called a spirit}		% M. Tu Ja	2017
	\item	\tit{nchyang}	\tab	\tit{0271}	\quad	\sqt{The three servants}				% Htu Bu	2017
	\item	\tit{nga}		\tab	\tit{0272}	\quad	\sqt{The thief who stole cattle}			% Htu Bu	2017
	\item	\tit{shanngayi}	\tab	\tit{0263}	\quad	\sqt{The deer that lost its horn}			% Htu Bu	2017
\end{itemize}


% ------------------------------------------------------------------------------------------------------------------------------------------------------------------------------ %
% ------------------------------------------------------------------------------------------------------------------------------------------------------------------------------ %

\subsection{Kalamang}
\label{ssec:corpus-kalamang}

\noindent\tit{Eline Visser}

\begin{description}[labelwidth=6.5em,itemindent=0em,itemsep=0.25mm]
	\TabPositions{2em}
	\raggedright\small
	\item[glottocode]		\texttt{kara1499}
	\item[affiliation]		Papuan, West Bomberai
	\item[area spoken]		West Papua, Indonesia
	\item[varieties rec'd]	Maas and Antalisa
	\item[text types]		traditional narratives
	\item[sources]		\bcite{Visser2020}
	\medskip
	\item[identifier]		\texttt{kalamang}
	\item[availability]		since June 2021, version \texttt{2106}
	\item[GRAID]		7.0		\tab	{\small(\texttt{\tgeq\ths2106})}
	\item[RefIND]		\checkyes{}	\tab	{\small(\texttt{\tgeq\ths2106})}
	\item[ISNRef]		\checkyes{}	\tab	{\small(\texttt{\tgeq\ths2106})}
	\item[citation]		\hyperref[ssec:references-mc]{Visser, Eline. 2021. Multi-CAST Kalamang. In Haig, Geoffrey \& Schnell, Stefan (eds.), \tit{Multi-CAST: Multilingual corpus of annotated spoken texts}. (\turl{}{multicast.aspra.uni-bamberg.de/\#kalamang})} \nocite{Visser2021}
\end{description}

\noindent Kalamang is a Papuan language spoken on the Karas Islands in West Papua, Indonesia. It is spoken by some 130 people in two villages on the biggest of the Karas Islands: Maas and Antalisa. Kalamang is under pressure from the local lingua franca, a variant of Papuan Malay, and is not currently spoken by people born after 1990. The texts in this corpus are all traditional narratives and were recorded in 2018 and 2019 as part of Eline Visser's PhD project at Lund University in Sweden, which resulted in a comprehensive grammar of Kalamang \pcite{Visser2020}. All Kalamang linguistic and cultural data have been deposited on the Humanities Lab corpus server at Lund University.\footnote{\turl{http://hdl.handle.net/10050/00-0000-0000-0003-C3E8-1}{hdl.handle.net/10050/00-0000-0000-0003-C3E8-1}}


% ------------------------------------------------------------------------------------------------------------------------------------------------------------------------------ %
% ------------------------------------------------------------------------------------------------------------------------------------------------------------------------------ %

\subsection{Mandarin}
\label{ssec:corpus-mandarin}

\noindent\tit{Maria Vollmer}

\begin{description}[labelwidth=6.5em,itemindent=0em,itemsep=0.25mm]
	\TabPositions{2em}
	\raggedright\small
	\item[glottocode]		\texttt{mand1415}
	\item[affiliation]		Sino-Tibetan, Sinitic
	\item[area spoken]		People's Republic of China
	\item[varieties rec'd]	Pǔtōnghuà, Xī'ān and Dōngběi
	\item[text types]		traditional narratives
	\medskip
	\item[identifier]		\texttt{mandarin}
	\item[availability]		since January 2020, version \texttt{2001}
	\item[GRAID]		7.0		\tab	{\small(\texttt{\tgeq\ths2001})}
	\item[RefIND]		\checkyes{}	\tab	{\small(\texttt{\tgeq\ths2001})}
	\item[ISNRef]		\checkyes{}	\tab	{\small(\texttt{\tgeq\ths2001})}
	\item[citation]		\hyperref[ssec:references-mc]{Vollmer, Maria C. 2020. Multi-CAST Mandarin. In Haig, Geoffrey \& Schnell, Stefan (eds.), \tit{Multi-CAST: Multilingual corpus of annotated spoken texts}. (\tcode{multicast.aspra.uni-bamberg.de/\#mandarin})} \nocite{Vollmer2020}
\end{description}

\noindent The Multi-CAST Mandarin corpus consists of traditional narratives from three native speakers of Modern Standard Mandarin (MSM, officially referred to as Pǔtōnghuà, \sqt{common speech}). Standard Mandarin is in many ways an artificial construct; an idealized form of the language has been taught to children in schools nationwide, but actual usage remains highly influenced by regional languages. The narratives in the corpus were recorded in Xī'ān in Northwest China; two of the speakers are originally from Northeast China (Dōngběi), the third hails from Xī'ān. 

The recordings were made by Maria Vollmer during an exchange semester in 2015 and 2016, transcribed by Liu Ruoyu in 2016 and 2017 under the direction of Maria Vollmer, and subsequently translated, glossed, and annotated with GRAID between 2016 and 2019 by Maria Vollmer. Annotations with RefLex and ISNRef were added by Maria Vollmer and Adrian Kuqi in 2019. Further stories have been recorded and transcribed and are planned to be added to the corpus in the future.



% ------------------------------------------------------------------------------------------------------------------------------------------------------------------------------ %
% ------------------------------------------------------------------------------------------------------------------------------------------------------------------------------ %

\subsection{Matukar Panau}
\label{ssec:corpus-matukar}

\noindent\tit{Danielle Barth, Kira Davey, Maria Matheas}

\begin{description}[labelwidth=6.5em,itemindent=0em,itemsep=0.25mm]
	\TabPositions{2em}
	\raggedright\small
	\item[glottocode]		\texttt{matu1261}
	\item[affiliation]		Austronesian, Malayo-Polynesian, Oceanic
	\item[area spoken]		Matukar village, Madang Province, Papua New Guinea
	\item[varieties rec'd]	Matukar Panau
	\item[text types]		traditional narratives,\\autobiographical narratives
%	\item[sources]		\bcite{Forker2020}
	\medskip
	\item[identifier]		\texttt{matukar}
	\item[availability]		since November 2023, version \texttt{2311}
	\item[GRAID]		8.0		\tab	{\small(\texttt{\tgeq\ths2311})}
	\item[RefIND]		\checkyes{}	\tab	{\small(\texttt{\tgeq\ths2311})}
	\item[ISNRef]		\checkyes{}	\tab	{\small(\texttt{\tgeq\ths2311})}
	\item[citation]		\hyperref[ssec:references-mc]{Barth, Danielle \& Davey, Kira \& Matheas, Maria. Multi-CAST Matukar Panau. In Haig, Geoffrey \& Schnell, Stefan (eds.), \tit{Multi-CAST: Multilingual corpus of annotated spoken texts}. (\turl{}{multicast.aspra.uni-bamberg.de/\#matukar})} \nocite{Barth.Davey.Matheas2023}
\end{description}

\noindent Matukar Panau is a highly endangered Oceanic language spoken around 45 km north of Madang, Papua New Guinea. Although most children are no longer learning Matukar Panau, current speakers (approximately 300) form a vibrant community of multilinguals in dense social networks. As an Oceanic language on the Papua New Guinea coast, Matukar Panau has many interesting Papuan features.

The Multi-CAST Matukar Panau corpus constitutes a small subset of recordings made by Danielle Barth during her fieldwork between 2010\tnd2020;\footnote{Australian National University Asia-Pacific Innovation Program Grant, \tit{Resolving Ambiguity: What face-to-face communication can contribute} (PI: Danielle Barth).} language documentation is ongoing. Data has been transcribed and translated with help from local community members, especially Kadagoi Rawad Forepiso and Rudolf Raward. Recordings can be found in the ELAR and PARADISEC archives.\footnote{\turl{http://hdl.handle.net/2196/00-0000-0000-0012-388F-3}{hdl.handle.net/2196/00-0000-0000-0012-388F-3}}\footnote{\turl{http://catalog.paradisec.org.au/collections/DGB1}{catalog.paradisec.org.au/collections/DGB1}; \turl{http://catalog.paradisec.org.au/collections/SocCog}{catalog.paradisec.org.au/collections/SocCog}} More information and resources on the language can be found on the project website.\footnote{\turl{https://matukar.wordpress.com/}{matukar.wordpress.com}}

The Multi-CAST texts were glossed with GRAID, RefIND, and ISNRef by Danielle Barth, Kira Davey and Maria Matheas. In addition to monologue narratives, some stimulus-based conversational descriptions have also been annotated with these schemata to enable research about referent expression when describing familiar and unfamiliar objects, places, and people. Recordings of these events are archived in ELAR and PARADISEC and those archives will eventually also provide open access to ELAN files with the annotations.


% - - - - - - - - - - - - - - - - - - - - - - - - - - - - - - - - - - - - - - - - - - - - - - - - - - - - - - - - - - - - - - - - - - - - - - - - - - - - - - - - - - - - - - - - - - - - - - - - - - - - %
% - - - - - - - - - - - - - - - - - - - - - - - - - - - - - - - - - - - - - - - - - - - - - - - - - - - - - - - - - - - - - - - - - - - - - - - - - - - - - - - - - - - - - - - - - - - - - - - - - - - - %

\subsubsection*{Background to the recordings}

\paragraph{bklife}
Speaker MP01. This autobiographical narration was recording in 2013 by MP01 in Surumarang, Papua New Guinea. It was recorded directly after MP01 recorded the text \tit{mariu}. He recounts the story of how his mother and father met, how his father spent time in jail, how MP01 married and had children, and the kind of work he does in the village.

\paragraph{fishing}
Speaker MP02. In this recording the speaker is at MP07's house in Matukar with her friend. She describes her fishing activity the day before and her plans for the rest of the day and the next.

\paragraph{kadagoi}
Speaker MP03. This narration was recorded in 2013 in Matukar. MP03 tells a short story about where her parents are from, about her husband, and about how her family lives their life in the village.

\paragraph{manub}
Speaker MP04. This traditional story was recorded in Matukar in 2018 at MP07's house by MP04. The story features the brothers Manub and Kulbob who come to Matukar and set up their homes at Dador, an area of Matukar. Dador is both a place and a woman. Manub and Kulbob are characters that feature in several different important origin stories in various parts of Madang province.

\paragraph{mariu}
Speaker MP01. This and the next recording (\tit{niu} are retellings of the same traditional origin story with some slight variation in the details. This version was recorded in 2013 with MP01 in Surumarang. The story describes how coconuts (Matukar Panau \tword{niu}) came about through the head of a body of a woman. This version of the story also describes the source of betelnuts (Matukar Panau \tword{mariu}), lime powder, mustard stick, and tobacco as coming from the same woman's body.

\paragraph{niu}
Speaker MP05. Like \tit{mariu}, this is a retelling of the story of how coconuts came about, though in this version it is through the head of a body of a man. This text was recorded in 2018 with MP05 in Matukar at MP07's house.

\paragraph{ww2}
Speaker MP06. This narration was recorded in 2019 under the house of MP06's family. MP06 lives under the house due to reduced mobility. She is the oldest Matukar Panau speaker (known at time of recording). She recounts what World War 2 was like for her as a young girl, escaping into the bush when Japanese soldiers came. She also describes meeting Yali after the war was over. Yali is an important figure in Papua New Guinea history, especially in Madang province. He later became a religious figure.

\paragraph{yali}
Speaker MP07. This narration was recorded in 2019 in Matukar at the house of MP07. She describes what it was like for her to meet Yali, a religious figure with a large following in Madang at the time. She and her then-husband request Yali's help to get pregnant, and Yali gives them a blessed betelnut which allows MP07 to get pregnant with her son.



% ------------------------------------------------------------------------------------------------------------------------------------------------------------------------------ %
% ------------------------------------------------------------------------------------------------------------------------------------------------------------------------------ %

\subsection{Nafsan}
\label{ssec:corpus-nafsan}

\noindent\tit{Nick Thieberger, Timothy Brickell}

\begin{description}[labelwidth=6.5em,itemindent=0em,itemsep=0.25mm]
	\TabPositions{2em}
	\raggedright\small
	\item[glottocode]		\texttt{sout2856}
	\item[affiliation]		Austronesian, Malayo-Polynesian, Oceanic, Vanuatu, Central
	\item[area spoken]		Vanuatu, Central Vanuatu, Efate
	\item[varieties rec'd]	Efate
	\item[text types]		traditional narratives
	\item[sources]		\bcite{Thieberger2006}
	\medskip
	\item[identifier]		\texttt{nafsan}
	\item[availability]		since August 2019, version \texttt{1908}
	\item[GRAID]		7.0		\tab	{\small(\texttt{\tgeq\ths1908})}
	\item[RefIND]		\checkyes{}	\tab	{\small(\texttt{\tgeq\ths1908})}
	\item[ISNRef]		\checkyes{}	\tab	{\small(\texttt{\tgeq\ths1908})}
	\item[citation]		\hyperref[ssec:references-mc]{Thieberger, Nick \& Brickell, Timothy. 2019. Multi-CAST Nafsan. In Haig, Geoffrey \& Schnell, Stefan (eds.), \tit{Multi-CAST: Multilingual corpus of annotated spoken texts}. (\turl{}{multicast.aspra.uni-bamberg.de/\#nafsan})} \nocite{Thieberger.Brickell2019}
\end{description}

\noindent The Nafsan language, also known as South Efate, is a Southern Oceanic language spoken on the island of Efate in central Vanuatu. As of 2005, there are approximately 6\ths000 speakers of Nafsan living in coastal villages from Pango to Eton. A description of the language can be found in \tcite{Thieberger2006}.

The Multi-CAST Nafsan corpus constitutes a subset of the material collected by Nick Thieberger for his PhD research over three periods of fieldwork in the villages of Eratap and Erakor in South Efate between 1995 and 2000, and during subsequent trips. The entirety of the data has been archived in PARADISEC \pcite{Thieberger1995},\footnote{\turl{http://catalog.paradisec.org.au/collections/NT1}{catalog.paradisec.org.au/collections/NT1}} and can also be accessed via ANNIS.\footnote{\turl{https://gerlingo.com/language_detail.php?langID=6}{gerlingo.com/language\_detail.php?langID=6}} See further \tcite{Thieberger2004}. Session name correspondences with Multi-CAST are as follows:
%
\begin{itemize}
	\TabPositions{5em}
	\item	\tit{kori}	\tab	\tit{094}	\quad	\sqt{A devil at Nguna}
	\item	\tit{lelep}	\tab	\tit{092}	\quad	\sqt{Tabu stories}
	\item	\tit{lisau}	\tab	\tit{077}	\quad	\sqt{Lisau}
	\item	\tit{litog}	\tab	\tit{075}	\quad	\sqt{Litong}
	\item	\tit{maal}	\tab	\tit{024}	\quad	\sqt{The hawk and the owl}
	\item	\tit{nmatu}	\tab	\tit{013}	\quad	\sqt{The pig wife}
	\item	\tit{ntwam}	\tab	\tit{019}	\quad	\sqt{The devil pig}
	\item	\tit{taapes}	\tab	\tit{078}	\quad	\sqt{The chicken and the swamphen}
	\item	\tit{tafra}	\tab	\tit{023}	\quad	\sqt{A story of a whale}
\end{itemize}
%
The texts were glossed with GRAID by Nick Thieberger and Timothy Brickell, and subsequently annotated with RefIND by Adrian Kuqi under supervision of Stefan Schnell.


% ------------------------------------------------------------------------------------------------------------------------------------------------------------------------------ %
% ------------------------------------------------------------------------------------------------------------------------------------------------------------------------------ %

\subsection{Northern Kurdish}
\label{ssec:corpus-nkurd}

\noindent\tit{Geoffrey Haig, Maria Vollmer, Hanna Thiele}

\begin{description}[labelwidth=6.5em,itemindent=0em,itemsep=0.25mm]
	\TabPositions{2em}
	\raggedright\small
	\item[glottocode]		\texttt{nort2641}
	\item[affiliation]		Indo-European, Iranian, Northwestern
	\item[area spoken]		eastern Turkey; northern Iraq; western Iran
	\item[varieties rec'd]	Northern Kurmanji, Erzurum and Muş
	\item[text types]		traditional narratives
	\item[sources]		\bcite{Haig2018}; \bcite{Haig.Opengin2018}; \bcite{Opengin.Haig2014}
	\medskip
	\item[identifier]		\texttt{nkurd}
	\item[availability]		since May 2015, version \texttt{1505}
	\item[GRAID]		7.0		\tab	{\small(\texttt{\tgeq\ths1505})}
	\item[RefIND]		\checksome{}	\tab	{\small(\texttt{\tgeq\ths1907})}
	\item[ISNRef]		\checksome{}	\tab	{\small(\texttt{\tgeq\ths1907})}
	\item[citation]		\hyperref[ssec:references-mc]{Haig, Geoffrey \& Vollmer, Maria \& Thiele, Hanna. 2019. Multi-CAST Northern Kurdish. In Haig, Geoffrey \& Schnell, Stefan (eds.), \tit{Multi-CAST: Multilingual corpus of annotated spoken texts}. (\turl{}{multicast.aspra.uni-bamberg.de/\#nkurd})} \nocite{Haig.Vollmer.Thiele2019}
\end{description}

\noindent Northern Kurdish, also known as Kurmanjî, is a Northwest Iranian language spoken in eastern Turkey, Iraq, Syria, and parts of western Iran. The three texts recorded here are traditional narratives, from a female and a male speaker who grew up near the townships of Erzurum and Muş in eastern Turkey, respectively.

The texts were recorded in Germany in the 1990s and early 2000s, and subsequently transcribed, translated, and annotated by Geoffrey Haig, Abdullah Incekan, Hanna Thiele, and Maria Vollmer. A description of the language can be found in \tcite{Haig2018}.


% - - - - - - - - - - - - - - - - - - - - - - - - - - - - - - - - - - - - - - - - - - - - - - - - - - - - - - - - - - - - - - - - - - - - - - - - - - - - - - - - - - - - - - - - - - - - - - - - - - - - %
% - - - - - - - - - - - - - - - - - - - - - - - - - - - - - - - - - - - - - - - - - - - - - - - - - - - - - - - - - - - - - - - - - - - - - - - - - - - - - - - - - - - - - - - - - - - - - - - - - - - - %

\subsubsection*{Background to the recordings}

\paragraph{muserz01, muserz03}
These two texts were recorded by Geoffrey Haig with a speaker called Miheme (NK01), who grew up in a village near Muş. The speaker had left Turkey approximately ten years previously and had since settled in Germany. The recordings were made in Miheme's allotment garden in Kiel, North Germany, in the company of his wife and another friend of the family. Geoffrey Haig made a long series of recordings with Miheme, most of which have been transcribed and translated by Geoffrey Haig with the assistance of native speakers.

The stories are Miheme's renderings of traditional Kurdish folkloric texts. Although not a trained storyteller, Miheme relished the opportunity to tell these stories, most of which he was recalling from childhood memories. He had no qualms about embellishing them in various ways when his memory failed him. His Kurdish is quite strongly influenced by Turkish, his main language of communication over the past two decades, but he is undoubtedly a fluent speaker of Kurmanji.

\paragraph{muserz02}
This text was recorded by Abdullah Incekan in 2002 in Essen, Germany; the speaker is his grandmother Güllü Tunç (NK02), who was visiting Germany at the time. The atmosphere was relaxed; a number of family members including small children were present during the recordings. The speaker is a monolingual Kurmanji speaker who has spent her lifetime in a village of the region Tekman, south of Erzurum. The text was transcribed by Abdullah Incekan and Geoffrey Haig, and translated by Geoffrey Haig.

As regards content, this text is undubitably related to the well-known fairy tale Cinderella, and contains key motifs such as the evil stepmother, the slipper, the prince and so on, but the latter part of the story seems to stem from a different source, and at times the narrative lacks coherence.


% ------------------------------------------------------------------------------------------------------------------------------------------------------------------------------ %
% ------------------------------------------------------------------------------------------------------------------------------------------------------------------------------ %

\subsection{Persian}
\label{ssec:corpus-persian}

\noindent\tit{Shirin Adibifar}

\begin{description}[labelwidth=6.5em,itemindent=0em,itemsep=0.25mm]
	\TabPositions{2em}
	\raggedright\small
	\item[glottocode]		\texttt{tehr1242}
	\item[affiliation]		Indo-European, Iranian, Southwestern
	\item[area spoken]		Iran
	\item[varieties rec'd]	Farsi, Tehran and Sari
	\item[text types]		stimulus-based narratives
	\medskip
	\item[identifier]		\texttt{persian}
	\item[availability]		since June 2016, version \texttt{1606}
	\item[GRAID]		7.0		\tab	{\small(\texttt{\tgeq\ths1606})}
	\item[RefIND]		\checkno{}
	\item[ISNRef]		\checkno{}
	\item[citation]		\hyperref[ssec:references-mc]{Adibifar, Shirin. 2016. Multi-CAST Persian. In Haig, Geoffrey \& Schnell, Stefan (eds.), \tit{Multi-CAST: Multilingual corpus of annotated spoken texts}. (\turl{}{multicast.aspra.uni-bamberg.de/\#persian})} \nocite{Adibifar2016}
\end{description}

\noindent Persian is an Iranian language with official variants spoken in Iran, Afghanistan, and parts of Tajikistan; the variety spoken in Iran is also referred to as Farsi.

The texts in this corpus are narrative retellings of the Pear film (Chafe 1980), a roughly five minute-long short film about a boy stealing the fruit a man had been picking. The recordings were made by Shirin Adibifar in Tehran and locations in province of Mazandaran in 2015. In total, there are 29 recordings, each from a different native speaker of Persian, 17 of which are female and 12 male; the median age is 25, with a range of 20 to 39. All speakers have received at least some measure of university-level education.

Each text was produced in an interview-like setting, in which the corpus compiler (Adibifar) showed the speaker a 6-minute video (the \tit{Pear Story}, cf.\@ \bcite{Chafe1980}) on a Dell color laptop computer with a 14-inch screen. At the end of the film, the interviewees were asked to recount the events of the film in their own words. The instructions were given by the researcher in their native language, Persian, and each participant received the same set of instructions. The interval between speakers watching of the movie and retelling its contents was less than five minutes. The participants were also asked to provide basic information regarding their age, gender, level of education, places of socialization, languages of communication (inside and outside of domestic settings), the language(s) of their parents, as well as their contact information in case of further questions.

The first half of the recordings (with \tit{g1} in the text name) took place in a relaxed domestic setting in the interviewer's hometown in the province of Mazanderan in northern Iran, and in three cases, in the speakers' apartments in Tübingen, Germany. The remainder (with \tit{g2} in the text name) were conducted with students from the Islamic Azad University in Tehran and Behšahr University in Mazanderan Province in seminar rooms of the two universities.


% ------------------------------------------------------------------------------------------------------------------------------------------------------------------------------ %
% ------------------------------------------------------------------------------------------------------------------------------------------------------------------------------ %

\subsection{Sanzhi Dargwa}
\label{ssec:corpus-sanzhi}

\noindent\tit{Diana Forker, Nils Norman Schiborr}

\begin{description}[labelwidth=6.5em,itemindent=0em,itemsep=0.25mm]
	\TabPositions{2em}
	\raggedright\small
	\item[glottocode]		\texttt{sanz1248}
	\item[affiliation]		Nakh-Daghestanian (Caucasian), Dargwa, Southern Dargwa
	\item[area spoken]		Druzhba town, central Daghestan, Russia
	\item[varieties rec'd]	Sanzhi
	\item[text types]		traditional narratives,\\autobiographical narratives
	\item[sources]		\bcite{Forker2020}
	\medskip
	\item[identifier]		\texttt{sanzhi}
	\item[availability]		since May 2019, version \texttt{1905}
	\item[GRAID]		7.0		\tab	{\small(\texttt{\tgeq\ths1905})}
	\item[RefIND]		\checkyes{}	\tab	{\small(\texttt{\tgeq\ths1905})}
	\item[ISNRef]		\checkyes{}	\tab	{\small(\texttt{\tgeq\ths1905})}
	\item[citation]		\hyperref[ssec:references-mc]{Forker, Diana \& Schiborr, Nils N. 2019. Multi-CAST Sanzhi Dargwa. In Haig, Geoffrey \& Schnell, Stefan (eds.), \tit{Multi-CAST: Multilingual corpus of annotated spoken texts}. (\turl{}{multicast.aspra.uni-bamberg.de/\#sanzhi})} \nocite{Forker.Schiborr2019}
\end{description}

\noindent Sanzhi Dargwa is a Nakh-Daghestanian (Caucasian) language from the Dargwa subbranch. From 1968 onwards, over a relatively short span of time, all Sanzhi speakers left their village of Sanzhi in the mountains of central Daghestan, Russia, to move to linguistically and ethnically heterogeneous settlements in the lowlands, mostly to the town of Druzhba. Today Sanzhi is spoken by approximately 250 speakers, and heavily endangered.

The eight texts in this corpus comprise a small subset of the material that was recorded, transcribed, translated, and glossed by Diana Forker and other researchers with the assistance of Gadzhimurad Gadzhimuradov, a native speaker, as part of a DOBES language documentation project (2012–2019). The entirety of the data has been archived at the Language Archive of the MPI, and is available on request.\footnote{\turl{https://hdl.handle.net/1839/00-0000-0000-0018-A4D4-6}{hdl.handle.net/1839/00-0000-0000-0018-A4D4-6}} A subcorpus of around ten hours has been fully glossed and translated into Russian and English, and is freely accessible online.\footnote{\turl{http://web-corpora.net/SanzhiDargwaCorpus/search/index.php?interface_language=en}{web-corpora.net/SanzhiDargwaCorpus/search/}} Session name correspondences between the Language Archive and Multi-CAST are as follows:
%
\begin{itemize}
	\TabPositions{5em}
	\item	\tit{asabali}		\tab	\tit{Sanzhi\_04\_08\_2013\_DF\_003}
	\item	\tit{bazhuk}		\tab	\tit{Sanzhi\_03\_08\_2013\_DF\_001}
	\item	\tit{dragon}		\tab	\tit{Sanzhi\_03\_08\_2013\_DF\_002}
	\item	\tit{kurban}		\tab	\tit{Sanzhi\_26\_07\_2011\_RM\_005}
	\item	\tit{mill}		\tab	\tit{Sanzhi\_30\_08\_2013\_HM\_001}
	\item	\tit{patima}		\tab	\tit{Sanzhi\_19\_03\_2013\_DF\_001}
	\item	\tit{ramazan}	\tab	\tit{Sanzhi\_08\_08\_2012\_RMDF\_004}
	\item	\tit{tape}		\tab	\tit{Sanzhi\_26\_07\_2011\_RM\_010}
\end{itemize}
%
\tcite{Forker2020} is a comprehensive grammar of Sanzhi Dargwa compiled on the basis on the collected material. The texts chosen for Multi-CAST are a mixture of spontaneously produced autobiographical and traditional narratives. They were annotated for Multi-CAST by Nils Schiborr.


% - - - - - - - - - - - - - - - - - - - - - - - - - - - - - - - - - - - - - - - - - - - - - - - - - - - - - - - - - - - - - - - - - - - - - - - - - - - - - - - - - - - - - - - - - - - - - - - - - - - - %
% - - - - - - - - - - - - - - - - - - - - - - - - - - - - - - - - - - - - - - - - - - - - - - - - - - - - - - - - - - - - - - - - - - - - - - - - - - - - - - - - - - - - - - - - - - - - - - - - - - - - %

\subsubsection*{Background to the recordings}

\paragraph{asabali}
Speaker SD01. Recorded by Diana Forker in August 2013 in Druzhba, Daghestan, Russia. The autobiographical retelling of the speaker's years as a young man, working first as a guard in the army, then as a miner, and later as a bus driver for a local factory.

%\paragraph{barkalla}
%Speaker SD02. Recorded by Diana Forker in September 2014 in Druzhba, Daghestan, Russia. A traditional narrative in which a rich man, a fox, a snake, and a bear are saved from a hunter's pit by a pauper. The animals show their gratitude by rewarding their saviour generously; the rich man, despite promising great rewards, refuses to acknowledge ever having been in trouble.
%%
%This text is a naturalistic translation from Standard Dargwa.
%
%\noindent Corresponds to the session \tit{Sanzhi\_24\_09\_2014\_DF\_002} in The Language Archive of the MPI.

\paragraph{bazhuk}
Speaker SD02. Recorded by Diana Forker in August 2013 in Druzhba, Daghestan, Russia. A traditional narrative in which a young shepherd is abducted by a witch after eating from her apple trees. He manages to hide himself and kill her in her own cooking pot.

%\paragraph{devils}
%Speaker SD05. Recorded by Diana Forker in August 2012 in Druzhba, Daghestan, Russia. A traditional narrative in which three brothers set out to find their fortune. The youngest protects a village by murdering a family of devils, for which he is rewarded with wealth and a bride. In traditional fairytale fashion, his brothers return emptyhanded from their journeys.
%%
%This text is a naturalistic translation from Standard Dargwa.
%
%\noindent Corresponds to the session \tit{Sanzhi\_22\_08\_2012\_DF\_001} in The Language Archive of the MPI.

\paragraph{dragon}
Speaker SD02. Recorded by Diana Forker in August 2013 in Druzhba, Daghestan, Russia. A traditional narrative in which a precocious young girl with a healthy appetite devours everyone in her village. Her brother, away on work, refuses to believe the rumours about her, and returns to the village only to be chased up a tree by his ravenous little sister, who has turned into a giant, fire-spewing monster. He calls the village dogs on her, and she is torn to shreds.

%\paragraph{happy}
%Speaker SD02. Recorded by Diana Forker in September 2014 in Druzhba, Daghestan, Russia. A traditional narrative in which two friends set out to find happiness: one by finding a hundred goats, the other by finding a hundred friends. When the former's goats are stolen in the night, the latter and his new friends help him out.
%%
%This text is a naturalistic translation from Standard Dargwa.
%
%\noindent Corresponds to the session \tit{Sanzhi\_24\_09\_2014\_DF\_001} in The Language Archive of the MPI.

\paragraph{kurban}
Speaker SD03. Recorded by Rasul Mutalov in July 2011 in Druzhba, Daghestan, Russia. An autobiographical narrative in which the speaker recounts the story of him and a friend playing a trick on the speaker's cousin, who desperately desires to become the head of a village \tnd\ despite being highly unqualified \tnd\ and will go to great lengths to get the appointment.

\paragraph{mill}
Speaker SD01. Recorded by Gadzhimurad Gadzhimuradov in August 2013 in Druzhba, Daghestan, Russia. Two traditional narratives, the first of which explains the significance of a particular mountain peak to the village of Sanzhi, the second of which humorously relates the story of the Sanzhi people's early troubles with watermills.
%
Only the second, longer narrative has been annotated with RefIND.

\paragraph{patima}
Speaker SD02. Recorded by Diana Forker in March 2013 in Druzhba, Daghestan, Russia. A traditional narrative in which a girl, Patima, goes into the forest to gather nuts for her sisters, only to find on her return that they have been eaten by a wolf. With the help of a sympathetic fox, she manages to kill the wolf and rescue her siblings from its gut.

\paragraph{ramazan}
Speaker SD04. Recorded by Diana Forker and Rasul Mutalov in August 2012 in Druzhba, Daghestan, Russia. The autobiographical recollections of the speaker in his three decades of work as a long-distance lorry driver. He heaps much praise on the Baltic countries, but has less favourable things to say about other places.

\paragraph{tape}
Speaker SD03. Recorded by Rasul Mutalov in July 2011 in Druzhba, Daghestan, Russia. An autobiographical narrative in which the speaker and a friend visit a shop that sells household goods. There they manage to get into an argument about which of them talks too much, inebriated or sober.



% ------------------------------------------------------------------------------------------------------------------------------------------------------------------------------ %
% ------------------------------------------------------------------------------------------------------------------------------------------------------------------------------ %

\subsection{Sumbawa}
\label{ssec:corpus-sumbawa}

\noindent\tit{Asako Shiohara}

\begin{description}[labelwidth=6.5em,itemindent=0em,itemsep=0.25mm]
	\TabPositions{2em}
	\raggedright\small
	\item[glottocode]		\texttt{sumb1241}
	\item[affiliation]		Austronesian, Malayo-Polynesian, Bali-Sasak-Sumbawa
	\item[area spoken]		Sumbawa island, Indonesia
	\item[varieties rec'd]	Sumbawa Besar
	\item[text types]		traditional narratives
	\item[sources]		\bcite{Shiohara2006}, \bcite{Shiohara2018}
	\medskip
	\item[identifier]		\texttt{sumbawa}
	\item[availability]		since November 2022, version \texttt{2211}
	\item[GRAID]		8.0		\tab	{\small(\texttt{\tgeq\ths2211})}
	\item[RefIND]		\checkyes{}	\tab	{\small(\texttt{\tgeq\ths2211})}
	\item[ISNRef]		\checkyes{}	\tab	{\small(\texttt{\tgeq\ths2211})}
	\item[citation]		\hyperref[ssec:references-mc]{Shiohara, Asako. 2022. Multi-CAST Sumbawa. In Haig, Geoffrey \& Schnell, Stefan (eds.), \tit{Multi-CAST: Multilingual corpus of annotated spoken texts}. (\turl{}{multicast.aspra.uni-bamberg.de/\#sumbawa})} \nocite{Shiohara2022}
\end{description}

\noindent Sumbawa (indigenous designation: Samawa) is a Western Austronesian language spoken in the western part of Sumbawa Island, Indonesia. Administratively, the area belongs to two districts, namely Sumbawa district (\tword{Kabupaten Sumbawa}) and West Sumbawa district (\tword{Kabupaten Sumbawa Barat}), in the province of West Nusa Tenggara (\tword{Nusa Tenggara Barat}). Sumbawa belongs to the Bali-Sasak-Sumbawa subgroup of the Malayo-Polynesian branch of the Austronesian language family (\bcite{Adelaar2005}; \bcite{Mbete1990}).

The corpus comprises the following five folktales:

\begin{itemize}
	\TabPositions{5em}
	\item	\tit{flood}		\tab	\tit{Tutir selunas utang}~~\sqt{A Story about Paying off Debt},\\\hspace*{1em}\tab told by Haji Renco (SB01)
	\item	\tit{kerekkure}	\tab	\tit{Tutir Lalu Kerekkure}~~\sqt{The Story of Prince Kerekkure},\\\hspace*{1em}\tab told by Agung Patawari (SB02)
	\item	\tit{langlelo}		\tab	\tit{Tutir Batu Langlelo}~~\sqt{The Story of the Langlelo Stone},\\\hspace*{1em}\tab told by Agung Patawari (SB02)
	\item	\tit{menangis}	\tab	\tit{Tanjung Menangis}~~\sqt{The Weeping Cape},\\\hspace*{1em}\tab told by Sulastri (SB03)
	\item	\tit{nuntut}		\tab	\tit{Tutir Lanang Mate}~~\sqt{The Story of Lanang Mate},\\\hspace*{1em}\tab told by Herman (SB04)
\end{itemize}
%
They were all collected by Asako Shiohara in 1996 and 1997. The first story was collected in Desa Bantu, a village close to the small town of Empang, where the other four stories were recorded. Among the several dialects of the Sumbawa language, the dialect spoken in the two locations is classified as the Sumbawa Besar dialect, which is distributed across a large part of the western Sumbawa-speaking area (\bcite{Mahsun1999}).
%
The texts were annotated for Multi-CAST by Shiohara between 2018 and 2022, with RefIND annotations added in 2022 by Tai Hong.


% ------------------------------------------------------------------------------------------------------------------------------------------------------------------------------ %
% ------------------------------------------------------------------------------------------------------------------------------------------------------------------------------ %

\subsection{Tabasaran}
\label{ssec:corpus-tabasaran}

\noindent\tit{Natalia Bogomolova, Dmitry Ganenkov, Nils Norman Schiborr}

\begin{description}[labelwidth=6.5em,itemindent=0em,itemsep=0.25mm]
	\TabPositions{2em}
	\raggedright\small
	\item[glottocode]		\texttt{taba1259}
	\item[affiliation]		Nakh-Daghestanian (Caucasian), Lezgic, Eastern Samur
	\item[area spoken]		Tabasaranksy District, central Daghestan, Russia
	\item[varieties rec'd]	Tabasaran
	\item[text types]		traditional narratives
%	\item[sources]		\bcite{((???))}
	\medskip
	\item[identifier]		\texttt{tabasaran}
	\item[availability]		since January 2021, version \texttt{2101}
	\item[GRAID]		7.0		\tab	{\small(\texttt{\tgeq\ths2021})}
	\item[RefIND]		\checkyes{}	\tab	{\small(\texttt{\tgeq\ths2021})}
	\item[ISNRef]		\checkyes{}	\tab	{\small(\texttt{\tgeq\ths2021})}
	\item[citation]		\hyperref[ssec:references-mc]{Bogomolova, Natalia \& Ganenkov, Dmitry \& Schiborr, Nils N. 2021. Multi-CAST Tabasaran. In Haig, Geoffrey \& Schnell, Stefan (eds.), \tit{Multi-CAST: Multilingual corpus of annotated spoken texts}. (\turl{}{multicast.aspra.uni-bamberg.de/\#tabasaran})} \nocite{Bogomolova.Ganenkov.Schiborr2021}
\end{description}

\noindent Tabasaran is a Nakh-Daghestanian (Caucasian) language from the Lezgic subbranch spoken in the Caucasus Mountains, in the Republic of Daghestan, Russia.
%
Recent census data puts the number of speakers at about \num{120000}; \tcite{Campbell.etal2017} classify the language as vulnerable.

The texts were recorded by Natalia Bogomolova with the assistance of Dmitry Ganenkov in 2010, and subsequently transcribed, glossed, and translated by Natalia Bogomolova. The annotations with GRAID and RefIND were added by Nils Schiborr between 2019 and 2020.
%
The five texts in this corpus are a mixture of traditional narratives and biographical texts.


% - - - - - - - - - - - - - - - - - - - - - - - - - - - - - - - - - - - - - - - - - - - - - - - - - - - - - - - - - - - - - - - - - - - - - - - - - - - - - - - - - - - - - - - - - - - - - - - - - - - - %
% - - - - - - - - - - - - - - - - - - - - - - - - - - - - - - - - - - - - - - - - - - - - - - - - - - - - - - - - - - - - - - - - - - - - - - - - - - - - - - - - - - - - - - - - - - - - - - - - - - - - %

\subsubsection*{Background to the recordings}

\paragraph{belt}
Speaker TS01. The story of the theft of a silver belt and the resulting dispute between two villages, which is cleverly solved by a pair of good friends.

\paragraph{horse}
Speaker TS02. A traditional narrative about three brothers, a vengeful magic horse, and the youngest brother's slow rise to wealth and recognition through wits and bravery.

\paragraph{naz}
Speaker TS01. A humorous tale of three brothers who could not be any more different from one another: The first is honourable, the second a scoundrel, the last a layabout.

\paragraph{nuradin}
Speaker TS01. A biographical retelling of the life of a tailor famous in the village of the speaker and beyond.

\paragraph{work}
Speaker TS01. A traditional tale about three brothers on the brink of destitution, who one by one set out to find work only to be tricked and killed by a wealthy and ruthless man. Only the youngest brother manages to outwit his employer and take his wealth for his own.



% ------------------------------------------------------------------------------------------------------------------------------------------------------------------------------ %
% ------------------------------------------------------------------------------------------------------------------------------------------------------------------------------ %

\subsection{Teop}
\label{ssec:corpus-teop}

\noindent\tit{Ulrike Mosel, Stefan Schnell}

\begin{description}[labelwidth=6.5em,itemindent=0em,itemsep=0.25mm]
	\TabPositions{2em}
	\raggedright\small
	\item[glottocode]		\texttt{teop1238}
	\item[affiliation]		Austronesian, Malayo-Polynesian, Oceanic, Nehan-Bougainville
	\item[area spoken]		Papua New Guinea, Bougainville
	\item[varieties rec'd]	Teop island
	\item[text types]		traditional narratives
	\item[sources]		\bcite{Mosel.Thiesen2007}
	\medskip
	\item[identifier]		\texttt{teop}
	\item[availability]		since May 2015, version \texttt{1505}
	\item[GRAID]		7.0		\tab	{\small(\texttt{\tgeq\ths1505})}
	\item[RefIND]		\checkyes{}	\tab	{\small(\texttt{\tgeq\ths1905})}
	\item[ISNRef]		\checkyes{}	\tab	{\small(\texttt{\tgeq\ths1905})}
	\item[citation]		\hyperref[ssec:references-mc]{Mosel, Ulrike \& Schnell, Stefan. 2015. Multi-CAST Teop. In Haig, Geoffrey \& Schnell, Stefan (eds.), \tit{Multi-CAST: Multilingual corpus of annotated spoken texts}. (\turl{}{multicast.aspra.uni-bamberg.de/\#teop})} \nocite{Mosel.Schnell2015}
\end{description}

\noindent Teop is an Oceanic language belonging to the Nehan-North Bougainville network of the North-West Solomonic group of the Meso-Melanesian cluster \pcite{Ross1988}. Accurate figures for the number of speakers are difficult to ascertain; figures from the last decade range from 5\ths000 to 10\ths000. The island of Bougainville was torn by a civil war which lasted from 1989 to 1998 and resulted in an estimated 18\ths000 to 20\ths000 casualities, with a devastating effect on the speech population. Factors like marriage outside of the Teop speech community, the pressure of neighbouring languages, the growing influence of Tok Pisin as a lingua franca, and the use of English in education all contribute to making Teop a highly endangered language.

The Teop texts were recorded by Ulrike Mosel and Enoch Horai Magum during the course of a DOBES language documentation project (principal
investigator: Ulrike Mosel) funded by the Volkswagen-Stiftung (grant no.\@ II 77 973). A sketch grammar of Teop \pcite{Mosel.Thiesen2007} and additional
materials are available on the DOBES website.\footnote{\url{dobes.mpi.nl/projects/teop/}} Another source of information on the meaning and construction of functional and content words is \tit{A multifunctional Teop-English dictionary} (MTED, \bcite{Mosel2019}).\footnote{\turl{https://dictionaria.clld.org/contributions/teop}{dictionaria.clld.org/contributions/teop\#twords}}

The texts were annotated with GRAID by Ulrike Mosel and Stefan Schnell. Referent indexing with RefIND was added in 2019 in a joint effort by Ulrike Mosel, Stefan Schnell, and Maria Vollmer.


% ------------------------------------------------------------------------------------------------------------------------------------------------------------------------------ %
% ------------------------------------------------------------------------------------------------------------------------------------------------------------------------------ %

\subsection{Tondano}
\label{ssec:corpus-tondano}

\noindent\tit{Timothy Brickell}

\begin{description}[labelwidth=6.5em,itemindent=0em,itemsep=0.25mm]
	\TabPositions{2em}
	\raggedright\small
	\item[glottocode]		\texttt{tond1251}
	\item[affiliation]		Austronesian, Malayo-Polynesian, Philippine, Minahasan, North, Northwest
	\item[area spoken]		Indonesia, North Sulawesi, Tondano town
	\item[varieties rec'd]	Toulour dialect
	\item[text types]		autobiographical narratives,\\stimulus-based narratives
	\item[sources]		\bcite{Sneddon1975, Brickell2015}
	\medskip
	\item[identifier]		\texttt{nkurd}
	\item[availability]		since June 2016, version \texttt{1606}
	\item[GRAID]		7.0		\tab	{\small(\texttt{\tgeq\ths1606})}
	\item[RefIND]		\checkno{}
	\item[ISNRef]		\checkno{}
	\item[citation]		\hyperref[ssec:references-mc]{Brickell, Timothy. 2016. Multi-CAST Tondano. In Haig, Geoffrey \& Schnell, Stefan (eds.), \tit{Multi-CAST: Multilingual corpus of annotated spoken texts}. (\turl{}{multicast.aspra.uni-bamberg.de/\#tondano})} \nocite{Brickell2016}
\end{description}

\noindent The Toulour dialect of Tondano is an Austronesian language spoken in and around the town of Tondano and the lake of the same name, and also in several villages to the east of this area. Tondano is located in the Minahasa regency on the northern tip of the island of Sulawesi, Indonesia. Current speaker numbers are difficult to ascertain, however earlier estimations of 70\ths000 \pcite[1]{Sneddon1975} and 91\ths000 \pcite{Wurm.Hattori1981} are now almost certainly incorrect. All Minahasan languages are endangered and have been shifting to the most commonly used language of wider communication, Manado Malay, since the early 20th century \pcite[299]{Wolff2010}. Anecdotal evidence and the personal experience of the researcher result in an upper range figure of 30\ths000 fluent speakers as being considered more accurate.

Tondano is not dominant in any domains of use, and is rarely used in everyday communication such as in workplaces, markets, or in the home. The last domain in which Tondano use remained strong was traditional agricultural work. However, with almost all remaining fluent Tondano speakers now aged 50 years and above, this situation is changing as speakers cease working in the fields. In contemporary society the language has little more than a token role in certain cultural settings such as church services, weddings, or occasionally speech contests in which people read from pre-prepared texts.

The only previous research on this language by a western academic was undertaken in 1975, the result of which was a phonology and sketch grammar \pcite{Sneddon1975} in the framework of Tagmemic grammar theory (as per \bcite{Longacre1960, Pike1964}). The sole contemporary linguistic research on is the PhD dissertation of \tcite{Brickell2015}. The data for this grammatical description come from various recording sessions which took place in North Sulawesi between 2011 and 2013 during three separate fieldtrips. These audio and video recording sessions all occurred at houses in Tondano township or at various locations closer to the lake. All the data were transcribed and translated in situ together with language consultants from within the Tondano speech community. There are approximately seven hours of recordings in total. All recordings are either monologues or dialogues which were \dqt{staged} in the sense of \tcite[185]{Himmelmann1998}, in that they took place predominantly for the purpose of the collection of primary linguistic data.

The data comprise a number of different recording genres. The first are instances where speakers narrated village and family history, or a specific culturally relevant story or event. The second are procedural narratives where speakers described how to carry out traditional indigenous activities (e.g.\@ cooking, or making handicrafts, or collecting flora and fauna) as they performed them. Finally, some narratives were elicited with the aid of visual stimuli (video recordings) whereby speakers watched and narrated as other community members performed these tasks. A number of dialogic texts were also recorded, but are not included in Multi-CAST.

Despite the staged nature of these communicative events, the recordings in the Tondano corpus are probably as natural as it is possible to be. Moreover, all data were recorded within the culture specific context of the indigenous Tondano speech community. All speakers who were recorded for the corpus gave informed consent for this data to be archived and accessed for further viewing and/or use. The research undertaken by Brickell in North Sulawesi was subject to the \tit{La Trobe University Human Research Ethics guidelines}.\footnote{\turl{https://www.latrobe.edu.au/__data/assets/pdf_file/0008/259217/Human-research-ethics-guidelines-may-2015.pdf}{latrobe.edu.au/\_\_data/assets/pdf\_file/0008/259217/\\\hquad Human-research-ethics-guidelines-may-2015.pdf}.} These guidelines are required to comply with the 2007 \tit{Australian National Statement on Ethical Conduct in Human Research}.\footnote{\turl{https://www.nhmrc.gov.au/about-us/publications/national-statement-ethical-conduct-human-research-2007-updated-2018}{nhmrc.gov.au/about-us/publications/\\\hquad national-statement-ethical-conduct-human-research-2007-updated-2018}.}


% - - - - - - - - - - - - - - - - - - - - - - - - - - - - - - - - - - - - - - - - - - - - - - - - - - - - - - - - - - - - - - - - - - - - - - - - - - - - - - - - - - - - - - - - - - - - - - - - - - - - %
% - - - - - - - - - - - - - - - - - - - - - - - - - - - - - - - - - - - - - - - - - - - - - - - - - - - - - - - - - - - - - - - - - - - - - - - - - - - - - - - - - - - - - - - - - - - - - - - - - - - - %

\subsubsection*{Background to the recordings}

\paragraph{gulamera, kiniar02}
These recording were taken in November 2011 and May 2013 at the houses of two speakers (TD01 and TD03) in the Rinegetan and Kiniar suburbs of Tondano town. The speakers narrate while watching an elicitation video which depicts a group of people collecting palm sugar sap from the sugar palm (\tit{arenga pinnata}) tree. The sap is then heated before being poured into coconut shells to be sold as palm sugar when it has cooled.

\paragraph{holiday}
In this recording the speaker (TD01) describes the experience of travelling to Australia and staying with her granddaughters in Sydney and Brisbane. She describes the things she did and places she saw while there. This narration was recorded in the Rinegetan suburb of Tondano town in September 2011.

\paragraph{kiniar01, kiniar03}
In these recordings the speakers (TD02 and TD03) narrate an elicitation video in which people buy fruit bats (commonly \tit{pteropus alecto} or \tit{chironax melancephalus}) from a marketplace. The process of preparing, cooking, and eating bat curry is then described. The recordings took place at two houses in the Kiniar neighbourhood of Tondano town in May 2013.

\paragraph{mapalus}
The speaker (TD04) in this recording session talks about an aspect of Minahasan culture known as \tit{mapalus,} which is the term for how community members traditionally work together for mutual assistance. She also speaks about her experience during a well known historical event called the Permesta rebellion in which some Minahasans fought against the Brawijaya regiment of the Indonesian National Army. This narration was recorded at a house in the Rinegetan suburb of Tondano town in September 2011.

\paragraph{water}
This recording session took place in the Rinegetan suburb of Tondano town in August 2011. The speaker (TD05), her mother, and her mother’s friend were all recorded on this day. The speaker is narrating an elicitation video which depicts the collecting, cooking, and eating of sago grubs (the larvae of the \tit{thynchophorus ferrungineus} beetle) from a sugar palm (\tit{arenga pinnata}) tree.

\paragraph{watulaney}
This narration was recorded in the lounge room of a house in Tataaran, a suburb just outside of Tondano town in September 2011. The speaker (TD06) is discussing her family history and the history of her village of Watulaney, which is located approximately 30 kilometres to the east of Tondano.


% ------------------------------------------------------------------------------------------------------------------------------------------------------------------------------ %
% ------------------------------------------------------------------------------------------------------------------------------------------------------------------------------ %

\subsection{Tulil}
\label{ssec:corpus-tulil}

\noindent\tit{Chenxi Meng}

\begin{description}[labelwidth=6.5em,itemindent=0em,itemsep=0.25mm]
	\TabPositions{2em}
	\raggedright\small
	\item[glottocode]		\texttt{taul1251}
	\item[affiliation]		Papuan, Taulil-Butam
	\item[area spoken]		East New Britain, Papua New Guinea
	\item[varieties rec'd]	Tulil
	\item[text types]		traditional narratives,\\autobiographical narratives
	\item[sources]		\bcite{Meng2018}
	\medskip
	\item[identifier]		\texttt{tulil}
	\item[availability]		since July 2019, version \texttt{1907}
	\item[GRAID]		7.0		\tab	{\small(\texttt{\tgeq\ths1907})}
	\item[RefIND]		\checkyes{}	\tab	{\small(\texttt{\tgeq\ths1907})}
	\item[ISNRef]		\checkyes{}	\tab	{\small(\texttt{\tgeq\ths1907})}
	\item[citation]		\hyperref[ssec:references-mc]{Meng, Chenxi. 2019. Multi-CAST Tulil. In Haig, Geoffrey \& Schnell, Stefan (eds.), \tit{Multi-CAST: Multilingual corpus of annotated spoken texts}. (\turl{}{multicast.aspra.uni-bamberg.de/\#tulil})} \nocite{Meng2019}
\end{description}

Tulil, also known as Taulil, is a Papuan language spoken in the East New Britain Province of Papua New Guinea. In 2000, Tulil was spoken by approximately 2\ths000 people spread out over four villages. The Tulil people are referred to by their neighbours as \dqt{Taulil}, while \dqt{Tulil} is the name they call themselves. The Tulil share their villages with the Butam people, whose language (Butam) is to be considered extinct after the last speaker died in 1938 \pcite{Laufer1959}. According to the oral history of the Tulil people, they along with the Butam migrated from the island of New Ireland Island to their current home on New Britain at some point in the past, before the arrival of the Tolai people in the area.

The six texts in this corpus comprise a subset of a larger collection of material that was recorded and transcribed by Chenxi Meng during two field trips to East New Britain in 2012 and 2015 for her PhD project, which has resulted in a comprehensive grammar of Tulil \pcite{Meng2018}. The entirety of the data has been deposited in PARADISEC \pcite{Meng2014};\footnote{\turl{http://catalog.paradisec.org.au/collections/CM2}{catalog.paradisec.org.au/collections/CM2}} session name correspondences with Multi-CAST are as follows:
%
\begin{itemize}
	\TabPositions{5em}
	\item	\tit{all1}	\tab	\tit{AL\_L1}
	\item	\tit{alrm}	\tab	\tit{AL\_RM}
	\item	\tit{jkpp}	\tab	\tit{JK\_PP}
	\item	\tit{lnsl}	\tab	\tit{LN\_SL}
	\item	\tit{lrdw}	\tab	\tit{LR\_DW}
	\item	\tit{sves}	\tab	\tit{SV\_ES}
\end{itemize}
%
The texts selected for Multi-CAST include both traditional and autobiographical narratives. Annotations with RefIND were added by Maria Vollmer.

% UPDATE %
%\clearpage


% ------------------------------------------------------------------------------------------------------------------------------------------------------------------------------ %
% ------------------------------------------------------------------------------------------------------------------------------------------------------------------------------ %

\subsection{Vera'a}
\label{ssec:corpus-veraa}

\noindent\tit{Stefan Schnell}

\begin{description}[labelwidth=6.5em,itemindent=0em,itemsep=0.25mm]
	\TabPositions{2em}
	\raggedright\small
	\item[glottocode]		\texttt{vera1241}
	\item[affiliation]		Austronesian, Malayo-Polynesian, Oceanic, Vanuatu
	\item[area spoken]		Vanuatu, Banks Islands, Vanua Lava
	\item[varieties rec'd]	Vera'a village
	\item[text types]		traditional narratives
	\item[sources]		\bcite{Schnell2010}, \ycite{Schnell2011}, \ycite{Schnell2016b}
	\medskip
	\item[identifier]		\texttt{veraa}
	\item[availability]		since May 2015, version \texttt{1505}
	\item[GRAID]		7.0		\tab	{\small(\texttt{\tgeq\ths1505})}
	\item[RefIND]		\checkyes{}	\tab	{\small(\texttt{\tgeq\ths1905})}
	\item[ISNRef]		\checkyes{}	\tab	{\small(\texttt{\tgeq\ths1905})}
	\item[citation]		\hyperref[ssec:references-mc]{Schnell, Stefan. 2015. Multi-CAST Vera'a. In Haig, Geoffrey \& Schnell, Stefan (eds.), \tit{Multi-CAST: Multilingual corpus of annotated spoken texts}. (\turl{}{multicast.aspra.uni-bamberg.de/\#veraa})} \nocite{Schnell2015}
\end{description}

\noindent The Vera'a language has around 450 speakers, 250 of which live in the village of Vera'a on Vanua Lava, Vanuatu, and the other 200 being scattered along the coastline reaching from Vera'a up to the northern shore. The language was initially researched by Catriona Hyslop in the late 1990s and Alexander François in the early 2000s (various publications by Francois deal with Vera'a). Since 2007, the language has been extensively documented by Stefan Schnell, first as part of a documentation project funded within the VolkswagenStiftung's DOBES language documentation programme, and since 2012 as part of Schnell's ASC-funded project on argument realization in discourse of diverse languages.

To date, several hours of video and audio recordings of speech events have been collected by Schnell and other reseachers.\footnote{\turl{http://dobes.mpi.nl/projects/vures_veraa/}{dobes.mpi.nl/projects/vures\_veraa/}} A large proportion of the data has been transcribed, hand-written transcriptions being undertaken by native speakers, and later entered into ELAN by Stefan Schnell, together with a translation into English. Some of the recorded narratives have later been edited by a speaker of Vera'a, Makson Vores, and published as a book \pcite{Vores.etal2012}.

All speech events recorded took place in the Vera'a community on Vanua Lava. Most of these comprise \dqt{staged events} in the sense of \tcite{Himmelmann1998}, that is they took place mainly for the sake of being recorded as part of the documentation project. Recordings of public events are also included in the Vera'a documentation, and these do not constitute staged events in the strict sense, though speakers were at all times informed about their being recorded. Nevertheless, the Vera'a corpus can be regarded as comprising a large set of
fairly natural speech data recorded within the indigenous cultural setting of the speech community. While most recordings were made by Stefan Schnell, Makson Vores collected several narratives in 2012 and 2013.

In addition to narratives and public events, procedural and descriptive texts were recorded. The latter comprise descriptions of plant and fish species. Both of these types of descriptions were recorded in dedicated sessions focussing on ethno-biologial aspects of the Vera'a language. In both sessions, speakers were asked to describe the respective plant or fish species, with a part of the plant or a picture card of the fish in front of them or in their hands.

The Vera'a subcorpus of Multi-CAST constitutes a relatively small portion of the entire Vera'a corpus. In addition to narratives, some of the plant and fish descriptions have been GRAID'ed and will be added to the corpus in the future, as will edited narratives in an effort to enable research into medium-related variation in argument expression.

Annotations with RefIND were added to the texts in 2019 by Stefan Schnell and Maria Vollmer.


%%%%%%%%%%%%%%%%%%%%%%%%%%%%%%%%%%%%%%%%%%%%%%%%%%%%%%%%%%%%%%%%%%
%%%%%%%%%%%%%%%%%%%%%%%%%%%%%%%%%%%%%%%%%%%%%%%%%%%%%%%%%%%%%%%%%%
